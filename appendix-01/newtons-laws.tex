\begin{longtable}{p{0.5\textwidth} p{0.5\textwidth}}
  \tablesubsection{Newton's Laws of Motion}\label{ssec:newtons-laws}
  &\\%this is for spacing of the \tablesubsection{} command
\end{longtable}

\textbf{First Law (law of inertia):}  An object at rest will remain at rest unless acted on by an unbalanced force. An object in motion continues in motion with the same speed and in the same direction unless acted upon by an unbalanced force. 

The corollary to this law is that a net force will either cause an object to leave rest or change the speed of an object.

\textbf{Second Law:} Acceleration is produced when a force acts on a mass. The greater the mass (of the object being accelerated) the greater the amount of force needed (to accelerate the object). The acceleration applied to the object is directly proportional to the net force applied, and it is inversely proportional to the mass of the object: \[ \vec{F} \propto \vec{a} \] and \[ \vec{a} \propto \frac{1}{m}. \]

Therefore, let us arrive at the following, which should be pretty familiar to you: \[ \vec{F} = m \vec{a}. \]

\textbf{Third Law:} For every action there is an equal and opposite reaction. 

Consider the rocket. The rocket's \textit{action} is to push down on the ground with the force of its powerful engines, and the \textit{reaction} is that the ground pushes the rocket upwards with an \textit{equal force}. 

For more information, visit \url{http://www.physicsclassroom.com/Physics-Tutorial/Newton-s-Laws}.
%%% Local Variables:
%%% mode: latex
%%% TeX-master: "../main"
%%% End: