\begin{tabular}{p{0.5\textwidth} p{0.5\textwidth}}
  \tablesubsection{Miscellaneous Angular Formul\ae}\label{ssec:angular-formulae}
  
  \(\vec{\omega} = 2\pi f\) & The quantity $\vec{\omega}$ is alternatively termed the angular frequency. A frequency $f$ is stated typically in Hertz (\si{\hertz}) but also common is revolutions per minute or second, rotations per minute and cycles per minute/second (these are all essentially equivalent terminology) \\
  \(fT=1\) & Frequency $f$ and period $T$ are inversely related. Frequency is typically measured in \si{\hertz} while period is measured in \si{\second} \\

  \notabene{One radian is the angle subtended at the centre of a circle by an arc equal in length to the radius of the circle. Thus, an angle $\theta$ in radians is given in terms of the arc length $\ell$ it subtends on a circle of radius $r$ by the equation $\theta=\frac{\ell}{r}$. Furthermore, \(\SI{1}{\revolution}=\SI{360}{\degree}=\SI{2\pi}{\radian}\)}
  \notabene{When converting from \textit{Degrees} to \textit{Radians}, use \(x\textrm{\space}\si{\degree}\times\frac{\pi}{180}\). When converting from \textit{Radians} to \textit{Degrees}, use \(x\textrm{\space}\si{\radian}\times\frac{180}{\pi}\)}
  \notabene{When converting from \textit{Revolutions per Minute} to \textit{Radians per Second}, use \(x\textrm{\space}\si{\revolution\per\minute}\times\frac{2\pi}{60}\si{\radian\per\second}\)}
\end{tabular}
%%% Local Variables:
%%% mode: latex
%%% TeX-master: "../main"
%%% End: