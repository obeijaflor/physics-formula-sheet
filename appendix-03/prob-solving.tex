\begin{longtable}{p{0.5\textwidth} p{0.5\textwidth}}
    \tablesection{Appendix III: List of Problem-Solving Strategies}
	\tablesubsection{General Physics Problem-Solving Strategy} 
    & \\%needed for formatting
\end{longtable}
\vspace{-1cm}
\par Excerpt from \url{http://www.people.fas.harvard.edu/~djmorin/chap1.pdf}

\begin{enumerate}
	\item \textbf{Draw a Diagram} In the diagram, be sure to clearly label all the relevant quantities (forces, lengths, masses, etc). Diagrams are absolutely critical in certain types of problems. For example, in problems involving free-body diagrams or relativistic kinematics, drawing a diagram can change a hopelessly complicated problem into a near-trivial one. And even in cases where diagrams aren't this crucial, they're invariable very helpful. A picture is definitely worth a thousand words (and even a few more, if you label things!).
    \item \textbf{Record Known \& Unknown Quantities} In a simple problem, you may just do this in your head without realizing it. But in more difficult problems, it is very useful to explicitly write things out. For example, if there are three unknowns that you are trying to find, but you've written down only two facts, then you know there must be another you're missing, so you can go searching for it. It might be a conservation law, or an \(F=ma\) equation, etc.
    \item \textbf{Solve Things Symbolically} If you are solving a problem where the given quantities are specified numerically, you should immediately change the numbers to letters and solve the problem in terms of the letters. After you obtain an answer in terms of the letters, you can plug in the actual numerical values to obtain a numerical answer. That said, it should be noted that there are occasionally times when things get a bit messy when working with letters. For example, solving a system of three equations in three unknowns might be rather cumbersome unless you plug in the actual numbers. But in the vast majority of problems, it is highly advantageous to work entirely with letters.
    \item \textbf{Consider Units \& Dimensions} The units, or dimensions, of a quantity are the powers of mass, length, and time associated with it. For example, the units of speed are length per time. The consideration of units offers two main benefits. First, looking at units before you start a problem can tell you roughtly what the answer has to look like, up to numerical factors. Second, checking units at the end of a calculation (which is something you should \textit{always} do) can tell you if your answer has a chance at being correct. It won't tell you that your answer is definitely correct, but it might tell you that your answer is definitely incorrect. For example, if your goal in a problem is to find a length, and you end up with a mass, then you know it's time to look back over your work.
    \item \textbf{Check Limiting \& Special Cases} As with units, the consideration of limiting cases (or special cases) offers two main benefits. First, it can help you get started on a problem. If you're having trouble figuring out how a given system behaves, then you can imagine making, for example, a certain length become very large or very small, and then you can see what happens to the behaviour. Having convinced yourself that the length actually affects the system in extreme cases (or perhaps you will discover that the length doesn't affect things at all, it will then be easier to understand how it affects the system in general, which will then make it easier to write down the relevant quantitative equations (conservation laws, \(F=ma\) equations, etc.), which will allow you to fully solve the problem. In short, modifying the various parameters and seeing the effects on the system can lead to an enormous amount of information. Second, as with checking units, checking limiting cases (or special cases) is something you should \textit{always} do at the end of a calculation. But as with checking units, it won't tell you that your answer is definitely correct, but it might tell you that your answer is definitely incorrect. It is generally true that your intuition about limiting cases is much better than your intuition about generic values of the parameters. you should use this fact to your advantage.
    \item \textbf{Use Your Brain} Whenever you end up with a numerical answer to a problem, be sure to do a sanity check to see if the number is reasonable. If you've calculated the distance along the ground that a car skids before it comes to rest, and if you've gotten answer of a kilometer or a millimeter, then you know you've probably done something wrong. Errors of this sort often come from forgetting some powers of $10$ (say when converting kilometers to meters) or from multiplying something instead of dividing (although you should be able to catch this by checking your units too).
\end{enumerate}