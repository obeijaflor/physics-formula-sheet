\begin{longtable}{p{0.5\textwidth} p{0.5\textwidth}}
  \tablesection{Chapter 14: Sound}
  \tablesubsection{The Speed of Sound}

  \(\vec{v}=\displaystyle\sqrt{\frac{B}{\rho}}\) & Yields the speed of sound $\vec{v}$ where $B$ is the bulk modulus of the fluid through which the sound is traveling and $\rho$ is the equilibrium density of the fluid \\
  \(B\equiv-\displaystyle\frac{\Delta P}{\frac{\Delta V}{V}}\) & Yields the Bulk Modulus of a fluid where $\Delta P$ is the change in pressure and $\frac{\Delta V}{V}$ is the resulting fractional change in volume \\
  \(\vec{v} = \displaystyle\sqrt{\frac{Y}{\rho}}\) & Yields the speed of a longitudinal wave in a solid rod where $Y$ is Young's modulus of the solid and $\rho$ is its density \\
  \(\vec{v} = \left(\SI{331}{\meter\per\second}\right)\displaystyle\sqrt{\frac{T}{\SI{273}{\kelvin}}}\) & The relationship between the speed of sound and temperature where \SI{331}{\meter\per\second} is the speed of sound in air \\

  \tablesubsection{Energy \& Intensity of Sound Waves}

  \(I\equiv\displaystyle\frac{1}{A}\frac{\Delta E}{\Delta t}\) & The average intensity $I$ of a wave with area $A$ and the rate of energy flow through the surface $\frac{\Delta E}{\Delta t}$ in \si{\watt\per\meter\squared} \\
  \(I\equiv\displaystyle\frac{\textrm{power}}{\textrm{area}}=\frac{P}{A}\) & Yields the intensity of a longitudinal wave \\ \\%for extra spacing
  \(\beta\equiv 10\log\displaystyle\left(\frac{I}{I_0}\right)\) & Yields the decibel level of the sound wave where $I_0=1.0\e{-12}$\si{\watt\per\meter\squared} is the reference of intensity, the sound intensity at the threshold of hearing \\

  \tablesubsection{Spherical \& Plane Waves}

  \(I = \displaystyle\frac{\textrm{average power}}{\textrm{area}}=\frac{P_{av}}{A}=\frac{P_{av}}{4\pi r^2}\) & Yields the intensity of a sound at distance $r$ from the source of the sound where the sound is emitted in a perfectly spherical manner \\
  \(\displaystyle\frac{I_1}{I_2}=\frac{r_1^2}{r_2^2}\) & The ratio of the intensities of sounds at two distances $r_1$ and $r_2$ \\
  \(\displaystyle f_o=f_s\left(\frac{\vec{v} + \vec{v}_o}{\vec{v} - \vec{v}_s}\right)\) & Yields the observed change in frequency due to relative motion between the observer and sound source where $\vec{v}$ is the speed of sound in the observational medium $\vec{v}_o$ is the velocity of the observer $\vec{v}_s$ is the velocity of the sound source and $f_s$ is the frequency of the sound generated by the source \\

  \notabene{As an observer approaches the source of a wave, there is an observed increase in frequency|a ``blue shift.''|As an observer recedes from the source of a wave, there is an observed decrease in frequency|a ``red shift.''|This is the \textit{Doppler Effect}}
  \notabene{See \textit{Appendix I} on page \pageref{fig:shock-waves} for information concerning shock waves and supersonic objects}

  \tablesubsection{Interference of Sound Waves}

  \begin{tabular}{l l}
    \(r_2-r_1=n\lambda\) & \(\left(n=0, 1, 2, \ldots\right)\)
  \end{tabular} & If the path difference $r_2-r_1$ is zero or some integer multiple of wavelengths, then constructive interference occurs \\
  \begin{tabular}{l l}
    \(r_2-r_1=\left(n+\frac{1}{2}\right)\lambda\) & \(\left(n=0, 1, 2, \ldots\right)\)
  \end{tabular} & If the path difference $r_2-r_1$ is $\frac{1}{2}$, $1\frac{1}{2}$, $2\frac{1}{2}$, etc. wavelengths, destructive interference occurs \\

  \tablesubsection{Standing Waves}

  \(d_{NN}=\frac{1}{2}\lambda\) & Yields the distance between adjacent nodes in a standing wave with wavelength $\lambda$ \\
  \(f_1=\displaystyle\frac{1}{2L}\sqrt{\frac{\vec{F}}{\mu}}\) & Yields the fundamental frequency $f_1$ of a wave on a string with speed $v=\sqrt{\frac{\vec{F}}{\mu}}$ \\
  \begin{tabular}{l l}
    \(f_n=nf_1=\displaystyle\frac{n}{2L}\sqrt{\frac{\vec{F}}{\mu}}\) & \(\left(n=1, 2, 3, \ldots\right)\)
  \end{tabular} & Yields the $n^{th}$ harmonic of a wave with fundamental frequency $f_1$ \\

  \tablesubsection{Standing Waves in Air Columns}

  \(f_n=n\displaystyle\frac{\vec{v}}{2L}=nf_1\) & Yields the $n^{th}$ harmonic of a sound wave in a pipe open at both ends where $\vec{v}$ is the speed of sound in air \\
  \(f_n=n\displaystyle\frac{\vec{v}}{4L}=nf_1\) & Yields the $n^{th}$ harmonic of a sound wave in a pipe open at one end where $\vec{v}$ is the speed of sound in air \\

  \tablesubsection{Beats}

  \(f_b=\abs{f_2-f_1}\) & Yields the beat frequency in \textit{beats per second} as the difference in frequency between two sources. See Figure \ref{fig:beat-frequency} in \textit{Appendix I} on page \pageref{fig:beat-frequency} for a helpful diagram explaining beat frequencies \\
\end{longtable}
%%% Local Variables:
%%% mode: latex
%%% TeX-master: "../main"
%%% End: