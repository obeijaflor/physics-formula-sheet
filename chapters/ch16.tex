\begin{longtable}{p{0.5\textwidth} p{0.5\textwidth}}
  \tablesection{Chapter 16: Electrical Energy \& Capacitance}
  \tablesubsection{Potential Difference \& Electric Potential}

  \(W=qE_x\Delta\vec{x}=\Delta KE\) & Yields the work done by the vector component $E_x$ of an electric field $\vec{E}$ as a charge $q$ is displaced by a distance of $\Delta\vec{x}$ \\
  \(\Delta PE=-W=-qE_x\Delta\vec{x}\) & Yields the change in electric potential energy \\
  \(\Delta V=V_B-V_A=\displaystyle\frac{\Delta PE}{q}\) & Yields the electric potential difference $\Delta V$ between points $A$ and $B$ as charge $q$ moves between them, measured in Voltz $\si{\volt}=\si{\joule\per\coulomb}$ \\
  \(\Delta V=-E_x\Delta\vec{x}\) & An alternate form of the equation yielding electric potential \\
  \(V=k_e\displaystyle\frac{q}{r}\) & Yields the electric potential created by a point charge $q$ \\
  \(PE=q_2V_1=k_e\displaystyle\frac{q_1q_2}{r}\) & yields the potential energy of a pair of charges. If the charges are the same sign, $PE$ is positive. If the charges are opposite signs, $PE$ is negative \\

  \tablesubsection{Potentials \& Charged Conductors}

  \(W=-\Delta PE=-q\Delta V\) & Relates work and electric potential. Due to this equation, no net work is required to move a charge between two points that are at the same electric potential (i.e., $W=0$ when $V_B=V_A$) \\
  \(\SI{1}{\electronvolt}=\SI{1.60e-19}{\coulomb\volt}=\SI{1.60e-19}{\joule}\) & The electron volt; the kinetic energy that an electron gains when accelerated through a potential difference of \SI{1}{\volt} \\

  \tablesubsection{Capacitors \& Capacitance}

  \(C\equiv\displaystyle\frac{Q}{\Delta V}\) & Yields the capacitance $C$ of a capacitor as a ratio of the magnitude of the charge on either conductor (plate) to the magnitude of the potential difference between the conductors (plates), measured in farads $\si{\farad}=\si{\coulomb\per\volt}$ \\
  \(C=\displaystyle\frac{Q}{\Delta V}=\frac{\sigma A}{Ed}=\frac{\sigma A}{\left(\frac{\sigma}{\epsilon_0}\right)d}=\epsilon_0\frac{A}{d}\) & Yields the capacitance of a parallel-plate capacitor where $A$ is the area of one of the plates, $d$ is the distance between the plates, $\sigma$ is the magnitude of the charge per unit area on each plate, and $\epsilon_0$ is the permittivity of free space \\
  \(Q=Q_1+Q_2\) & Yields the total charge $Q$ stored in two capacitors operating in parallel with maximum charges $Q_1$ and $Q_2$ respectively \\
  \(C_{eq} = \sum C\) & Yields the equivalent capacitance $C_{eq}$ of a parallel combination of capacitors where $C$ is the capacitance of an individual capacitor \\
  \(\displaystyle\frac{1}{C_{eq}}=\sum\frac{1}{C}\) & Yields the equivalent capacitance of a series combination of capacitors \\

  \notabene{The equivalent capacitance of a parallel combination of capacitors is larger than any of the individual capacitances}
  \notabene{The equivalent capacitance of a series combination of capacitors is smaller than any of the individual capacitances}

  \tablesubsection{Energy Stored in a Charged Capacitor}

  \(\Delta W=\Delta V\Delta Q\) & Yields the work $\Delta W$ required to move more charge $\Delta Q$ through the potential difference $\Delta V$ of a capacitor if the potential difference at any instant during the charging process is $\Delta V$ \\
  \(W=\frac{1}{2}Q\Delta V\) & Yields the total work required to charge a capacitor to final charge $Q$ with potential difference $\Delta V$ \\
  \(\textrm{Energy stored} = \frac{1}{2}Q\Delta V=\frac{1}{2}C\left(\Delta V\right)^2=\displaystyle\frac{Q^2}{2C}\) & Yields the charge stored in a capacitor \\

  \tablesubsection{Capacitors with Dielectrics}

  \(\Delta V_0=\displaystyle\frac{Q_0}{C_0}\) & Yields the potential difference across the capacitor plates for a parallel-plate capacitor of charge $Q_0$ and capacitance $C_0$ \\
  \(\Delta V=\displaystyle\frac{\Delta V_0}{\kappa}\) & Yields the electric potential across a capacitor described above if a dielectric is inserted between the plates where $\kappa$ is the dielectric constant, which is different for each material \\
  \(C=\kappa\epsilon_0\displaystyle\frac{A}{d}=\kappa C_0\) & Yields the capacitance for a parallel-plate capacitor when a dielectric completely fills the distance $d$ between the plates of the capacitor with area $A$ and dielectric constant $\kappa$ and initial capacitance $C_0$ \\

  \notabene{A dielectric is an insulating material, such as rubber, plastic or waxed paper. When a dielectric is inserted between the plates of a capacitor, the capacitance increases. If the dielectric completely fills the space between the plates, the capacitance is multiplied by the dielectric constant}
  \notabene{While the capacitance could be made very large by increasing the distance $d$ between the plates, for any given plate separation there is a maximum electric field that can be produced in the dielectric before it breaks down and begins to conduct. This is the \textit{dielectric strength}, and for air its value is about \SI{3e6}{\volt\per\meter}}

\end{longtable}
%%% Local Variables:
%%% mode: latex
%%% TeX-master: "../main"
%%% End: