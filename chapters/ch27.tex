\begin{longtable}{p{0.5\textwidth} p{0.5\textwidth}}
  \tablesection{Chapter 27: Quantum Physics}
  \tablesubsection{Blackbody Radiation \& Planck's Hypothesis}

  \(\lambda_{max}T=\SI{0.2898e-2}{\meter\kelvin}\) & \textit{Wien's displacement law} where $\lambda_{max}$ is the wavelength at which a curve describing the emission spectrum for an object and $T$ is the absolute temperature of the object emitting the radiation \\
  \(E_n=nhf\) & Planck's formula for blackbody radiation where $E_n$ is the discrete energy available to a quantized particle, $n$ is the quantum number, $f$ is the frequency of vibration of the resonator and $h=\SI{6.626e-34}{\joule\second}$ is Planck's constant \\

  \notabene{A \textit{blackbody} is an ideal system which absorbs all radiation incident on it. Like all objects, bkackbodies emit thermal radiation. As the temperature of the blackbody increases, the total amount of energy it emits increases. Also, with increasing temperature, the peak of the distribution shifts to shorter wavelengths. This shift obeys Wien's displacement law, described above}
  \notabene{Planck hypothesized that blackbody radiation was produced by submicroscopic charged oscillators, termed \textit{resonators}. These resonators were only allowed to have certain discrete energies $E_n$, as described above. Because the energy of each resonator can only have discrete values, that energy is said to be \textit{quantized}. Each discrete energy value represents a different \textit{quantum state}}

  \tablesubsection{The Photoelectric Effect \& The Particle Theory of Light}

  \notabene{The following formul\ae\space consider a system which consists of an emitter $E$ of photoelectrons which strike the collector $C$ which is charged by a circuit with a variable power supply. When $C$ is positively charged it collects photoelectrons, producing a current indicating the flow of charged from $E$ to $C$. When $C$ is negatively charged, it repels photoelectrons, the current significantly decreases, because only those electrons having a kinetic energy greater than the magnitude of $e\Delta V$ reach $C$ where $e$ is the charge on the electron. When $\Delta V$ is equal to or more negative than $-\Delta V_s$|the \textit{stopping potential}|no electrons reach $C$ and the current is zero}

  \(KE_{max}=e\Delta V_s\) & Yields the maximum kinetic energy of photoelectrons as related to the stopping potential $\Delta V_s$ where $e$ is the charge on the electron \\
  \(E=hf\) & Yields the energy of a photon of light frequency $f$ where $h$ is Planck's constant \\
  \(KE_{max}=hf-\phi\) & \textit{The Photoelectric Effect Equation}; yields the maximum kinetic energy for a liberated photoelectron (a photoelectron which has received all the energy $hf$ from a photon) where $\phi$ is the \textit{work function} of the metal from which the photoelectron originated, measured in electron volts \si{\electronvolt} \\
  \(\lambda_c=\displaystyle\frac{hc}{\phi}\) & Yields the cutoff wavelength $\lambda_c$|the wavelength below which no photoelectrons are emitted regardless of light intensity|for a material with work function $\phi$ where $c$ is the speed of light \\

  \notabene{A photoelectron is an electron emitted due to light incident on certain metallic surfaces}

  \tablesubsection{X-Rays}

  \(e\Delta V=hf_{max}=\displaystyle\frac{hc}{\lambda_{min}}\) & Relates the initial energy of the electron $e\Delta V$ to the energy of the released photon $hf_{max}$ where $e\Delta V$ is the energy of the electron after it has been accelerated through a potential difference of $\Delta V$ volts \\
 \(\lambda_{min}=\displaystyle\frac{hc}{e\Delta V}\) & Yields the shortest wavelength radiation that can be produced by a potential difference of $\Delta V$ \\

  \notabene{X-Rays are produced when high-speed electrons are suddenly slowed down, such as when a metal target is struck by electrons that have been accelerated through a potential difference of several thousand volts}

  \tablesubsection{X-Ray Diffraction by Crystals}

  \begin{tabular}{l l}
    \(2d\sin\theta=m\lambda\) & \(m=1,2,3,\ldots\)
  \end{tabular} & \textit{Bragg's Law}; Relates the difference of the distance traveled by one light beam diffracted by a plane of a crystal to the distance traveled by a parallel light beam diffraced by another plane of the crystal \\

  \notabene{The \"Angstr\"om $\SI{1}{\angstrom}=10^{-10}$\,\si{\meter} is frequently used to measure the wavelengths of x-rays}

  \tablesubsection{The Compton Effect}

  \(\Delta\lambda=\lambda-\lambda_0=\displaystyle\frac{h}{m_ec}\left(1-\cos\theta\right)\) & \textit{The Compton Shift Formula}; Yields the change in the wavelength of a photon deflected by an angle $\theta$ due to a collision with an electron. The quantity $\frac{h}{m_ec}$ is the \textit{Compton wavelength} and has a value of \SI{0.00243}{\nano\meter} \\

  \tablesubsection{The Dual Nature of Light \& Matter}

  \(E=hf=\frac{hc}{\lambda}\) & Yields the energy of a photon \\
  \(p=\frac{E}{c}=\frac{hc}{c\lambda}=\frac{h}{\lambda}\) & Yields the momentum of a photon \\
  \(\lambda=\frac{h}{p}=\frac{h}{mv}\) & Yields the \textit{de Broglie wavelength} of a particle, the wavelength of all particles with momentum $p$ where $p=mv$ \\
  \(f=\displaystyle\frac{E}{h}\) & Yields the frequency of matter waves, where $E=hf$ \\

  \notabene{\textit{The De Broglie Hypothesis} postulates that because photons have wave and particle characteristics, perhaps all forms of matter have both properties. The Davisson-Germer experiment in 1927 confirmed the hypothesis by showing that electrons scattering off crystals form a diffraction pattern. The regularly spaced planes of atoms in crystalline regions of a nickel target act as a diffraction grating for the electron matter waves}

  \tablesubsection{The Heisenberg Uncertainty Principle}

  \(\Delta x\Delta p_x\geq\displaystyle\frac{h}{4\pi}\) & \textit{The Heisenberg Uncertainty Principle}; if a measurement of the position of a particle is made with precision $\Delta x$ and a simultaneous measurement of linear momentum is made with precision $\Delta p_x$, the product of the two uncertainties can never be smaller than $\frac{h}{4\pi}$ \\
  \(\Delta E\Delta t\geq\displaystyle\frac{h}{4\pi}\) & Another form of the uncertainty relationship between the energy $E$ of a system and a finite time interval $\Delta t$ \\

  \notabene{Due to the uncertainty principle, it is physically impossible to measure simultaneously the exact position and exact linear momentum of a particle. If $\Delta x$ is very small, $\Delta p_x$ is very large and vice-versa}
\end{longtable}
%%% Local Variables:
%%% mode: latex
%%% TeX-master: "../main"
%%% End: