\begin{longtable}{p{0.475\textwidth} p{0.475\textwidth}}
  \tablesection{Chapter 2: Motion in One Dimension}
  \tablesubsection{General Formul\ae}
	
  \notabene{A \textit{vector} quantity has both magnitude and direction while a \textit{scalar} quantity can be completely specified by its magnitude, but has no direction. Displacement $\Delta\vec{x}$, velocity $\vec{v}$, and acceleration $\vec{a}$ are vector quantities. Temperature $T$ is an example of a scalar quantity.}
	
  \(\Delta\vec{x} \equiv x_f - x_i\) & The displacement $\Delta\vec{x}$ of an object is defined as its \textit{change in position} where $x_i$ is the initial position of the object and $x_f$ is the final position of the object. Throughout this sheet the indices $i$ and $f$ will stand for initial and final, respectively. Displacement is measured in meters \\
  \(d = \displaystyle\sqrt{\left(x_f - x_i\right)^2 + \left(y_f - y_i\right)^2}\) & The distance $d$ between two coordinates, measured in meters \\
  \(\vec{v}_{avg} \equiv \displaystyle\frac{\Delta\vec{x}}{\Delta t} = \frac{x_f - x_i}{t_f - t_i}\) & The average velocity $\vec{v}_{avg}$ during time interval $\Delta t$ with displacement $\Delta\vec{x}$, measured in meters per second \\
  \(\vec{v}_{ins} \equiv \displaystyle\lim_{\Delta t\to 0}\frac{\Delta\vec{x}}{\Delta t}\) & The instantaneous velocity $\vec{v}_{ins}$ is the limit of the average velocity $\vec{v}_{avg}$ as the time interval $\Delta t$ becomes infinitesimally small, measured in meters per second \\
  \(\vec{a}_{avg} \equiv \displaystyle\frac{\Delta v}{\Delta t} = \frac{v_f - v_i}{t_f - t_i}\) & The average acceleration $\vec{a}_{avg}$ during the time interval $\Delta t$ is the change in velocity $\Delta v$ across the time interval $\Delta t$, measured in meters per second per second \si{\meter\per\second\squared} \\
  \(\vec{a}_{ins} \equiv\displaystyle\lim_{\Delta t\to 0}\frac{\Delta v}{\Delta t}\) & The instantaneous acceleration $\vec{a}_{ins}$ is the limit of the average acceleration $\vec{a}_{avg}$ as the time interval $\Delta t$ approaches 0, measured in \si{\meter\per\second\squared} \\
	
  \tablesubsection{One-Dimensional Motion with Constant Acceleration}
	
  \(\vec{v}_f = \vec{v}_i + \vec{a}t\) & The final velocity $\vec{v}_f$ of an object with initial velocity $\vec{v}_i$ and constant acceleration $\vec{a}$ across the time interval $t$ \\
  \(\Delta\vec{x} = \frac{1}{2}\left(\vec{v}_i + \vec{v}_f\right)t\) & The displacement $\Delta\vec{x}$ of an object with constant acceleration \\
  \(\Delta\vec{x} = \vec{v}_it + \frac{1}{2}\vec{a}t^2\) & The displacement $\Delta\vec{x}$ of an object with constant acceleration $\vec{a}$ \\
  \(\vec{v}_f = \displaystyle\sqrt{\vec{v}_i^2 + 2\vec{a}\Delta\vec{x}}\) & The final velocity $\vec{v}_f$ of an object with constant acceleration $\vec{a}$ \\
	
  \tablesubsection{Freely Falling Objects}
	
  \(\vec{a} = g = \SI{9.80665}{\meter\per\second\squared}\) & The acceleration due to gravity $g$ at sea level on Earth \\
  \(t_{max} = \displaystyle\frac{\vec{v}_i}{g}\) & The amount of time $t_{max}$ it will take for an object to reach its maximum height assuming $\vec{v}_i$ is opposite $g$ \\
  \(y = y_i + \vec{v}_it + \frac{1}{2}gt^2\) & The position $y$ of any object across time interval $t$ assuming $\vec{v}_i$ is opposite $g$ \\
  \(y_{max} = y_i + \displaystyle\frac{\vec{v}_i^2}{2g}\) & The maximum height $y_{max}$ of an object assuming $\vec{v}_i$ is opposite $g$ \\
  \(t = \displaystyle\sqrt{\frac{2\Delta y}{g}}\) & The time taken $t$ for an object to be displaced $\Delta y$ meters in the $y$-direction due to $g$ \\

\notabene{The initial velocity in most problems involving freely-falling objects is \SI{0}{\meter\per\second}}
\end{longtable}
%%% Local Variables:
%%% mode: latex
%%% TeX-master: "../main"
%%% End: