\begin{longtable}{p{0.5\textwidth} p{0.5\textwidth}}
  \tablesection{Chapter 17: Current \& Resistance}
  \tablesubsection{Electric Current}

  \(I_{av}\equiv\displaystyle\frac{\Delta Q}{\Delta t}\) & Yields the average current $I_{av}$, the rate at which charge flows through a surface measured in Amperes $\si{\ampere}=\si{\coulomb\per\second}$ \\
  \(I=\displaystyle\lim_{\Delta t\to 0}I_{av}=\lim_{\Delta t\to 0}\frac{\Delta Q}{\Delta t}\) & Yields the instantaneous current $I$ as the time interval approaches 0 \\
  \(\Delta Q=\textrm{number of carriers}\times\textrm{charge per carrier}=\left(nA\Delta x\right)q\) & Yields the mobile charge $\Delta Q$ moving through a conductor of cross-sectional area $A$ with length $\Delta x$ and a number of charge carriers (i.e., protons and electrons) $n$ where $q$ is the charge on each carrier \\
  \(\Delta Q=\left(nAv_d\Delta t\right)q\) & An alternate form of the equation yielding mobile charge $\Delta Q$ where the distance covered by charge carriers during the time interval $\Delta t$ is $\Delta x=v_d\Delta t$ where $v_d$ is the drift speed \\
  \(I=\displaystyle\lim_{\Delta t\to 0}\frac{\Delta Q}{\Delta t}=nqv_dA\) & Relates mobile charge to instantaneous current $I$ \\

  \notabene{The direction of conventional current used in the book which provided the majority of information for this formula sheet is in the direction positive charges flow (i.e., opposite electron flow)}

  \tablesubsection{Resistance, Resistivity, \& Ohm's Law}

  \(R\equiv\displaystyle\frac{\Delta V}{I}\) & Yields the resistance $R$ in ohms \si{\ohm} \\
  \(\Delta V=IR\) & Relates voltage to current and resistance where $I\propto\Delta V$ where a voltage $\Delta V$ is applied across the ends of a metallic conductor \\
  \(R=\rho\displaystyle\frac{\ell}{A}\) & Yields the resistance of a material with length $\ell$ and cross-sectional area $A$ where $\rho$ is the resistivity of the material \\

  \notabene{\textit{Ohmic} materials have constant resistance over a wide range of voltages. \textit{Nonohmic} materials have a resistance which changes with voltage or current}

  \tablesubsection{Temperature Variation of Resistance}

  \(\rho=\rho_0\left[1+\alpha\left(T-T_0\right)\right]\) & Yields the resistivity $\rho$ over a limited temperature range where the resistivity of most metals increases linearly with temperature, $\alpha$ is the temperature coefficient of resistivity, and $\rho_0$ is the resistivity at a reference temperature $T_0$ \\
  \(R=R_0\left[1+\alpha\left(T-T_0\right)\right]\) & Yields the resistance $R$ over a limited temperature range due to $R=\rho\frac{\ell}{A}$ \\
  \(P=I\Delta V\) & Yields the power $P$ of a system with current $I$ and electric potential $\Delta V$ in Joules \si{\joule} \\
  \(P=I^2R=\displaystyle\frac{\Delta V^2}{R}\) & An alternate form of the power equation where $\Delta V=IR$ \\
\end{longtable}
%%% Local Variables:
%%% mode: latex
%%% TeX-master: "../main"
%%% End: