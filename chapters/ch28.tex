\begin{longtable}{p{0.5\textwidth} p{0.5\textwidth}}
  \tablesection{Chapter 28: Atomic Physics}
  \tablesubsection{Atomic Spectra}

  \(\displaystyle\frac{1}{\lambda}=R_H\left(\frac{1}{2^2}-\frac{1}{n^2}\right)\) & Yields the \textit{Balmer Series} for identifying wavelengths in emission spectra where $n$ may have integral values of $3,4,5,\ldots$ and $R_H=\SI{1.0973732e7}{\meter^{-1}}$ is the Rydberg constant \\
  \(\displaystyle\frac{1}{\lambda}=R_H\left(\frac{1}{m^2}-\frac{1}{n^2}\right)\) & \textit{The Rydberg Equation}; A combination of the Paschen \& Balmer series where $m$ and $n$ are positive integers and $n>m$ \\

  \tablesubsection{The Bohr Model}

  \(PE=k_e\displaystyle\frac{q_1q_2}{r}=k_e\frac{\left(-e\right)e}{r}=-k_e\frac{e^2}{r}\) & Yields the electrical potential energy of an atom about which the electron travels in a circular orbit of radius $r$ with orbital speed $v$ \\
  \(E=KE+PE=\frac{1}{2}m_e\vec{v}^2-k_e\displaystyle\frac{e^2}{r}=-\frac{k_ee^2}{2r}\) & Yields the total energy of the atom assuming the nucleus is at rest \\
  \begin{tabular}{l l}
    \(r_n=\displaystyle\frac{n^2\hslash}{m_ek_ee^2}\) & \(n=1,2,3,\ldots\)
  \end{tabular} & Yields the radius $r$ of the orbit of the electron about an atom. This equation is based on the assumption that the electron can exist only in certain orbits determined by the integer $n$. The orbit with the smallest radius, the \textit{Bohr radius} $a_0$ corresponds to $n=1$ and has value $a_0=\frac{\hslash^2}{mk_ee^2}=\SI{0.0529}{\nano\meter}$ where $\hslash=\frac{h}{2\pi}=\SI{1.05e-34}{\joule\second}$ is the reduced Planck constant\\

  \tablesubsection{Characteristic X-Rays}

  \(E_K=0m_eZ^2_{eff}\displaystyle\frac{k_e^2e^4}{2\hslash^2}=-Z^2_{eff}E_0\) & Yields an estimate of the energy in the $K$ shell (the innermost electron shell with $n=1$) where $Z_{eff}=\left(Z-1\right)e$ is the effective nuclear charge from the nucleus of an atom of atomic number $Z$ and $E_0$ is the ground-state energy \\
\end{longtable}
%%% Local Variables:
%%% mode: latex
%%% TeX-master: "../main"
%%% End: