\begin{longtable}{p{0.5\textwidth} p{0.5\textwidth}}
  \tablesection{Chapter 26: Relativity}
  \tablesubsection{Time Dilation}

  \(\Delta t_p=\displaystyle\frac{\textrm{distance traveled}}{\textrm{speed}}\) & Yields the proper time interval $\Delta t_p$ between two events \\
  \(\Delta t=\displaystyle\frac{\Delta t_p}{\sqrt{1-\frac{v^2}{c^2}}}=\gamma\Delta t_p\) & Yields the time interval $\Delta t$ between two events measured by an observer moving with respect to those events and another observer at rest with respect to the events, thus $\Delta t>\Delta t_p$ and the proper time interval is expanded or dilated by the factor $\gamma$. This effect is known as \textit{Time Dilation} \\
  \(\gamma=\displaystyle\frac{1}{\sqrt{1-\frac{v^2}{c^2}}}\) & Yields the factor by which the time interval is expanded or dilated due to special relativity \\

  \notabene{In relativistic mechanics there is no such thing as absolute length or absolute time. Events at different locations that are observed to occur simultaneously in one frame are not observed to be simultaneous in another frame moving uniformly past the first, thus \textit{length and time measurements depend on the frame of reference}}

  \tablesubsection{Length Contraction}

  \(L=\displaystyle\frac{L_p}{\gamma}=L_p\sqrt{1-\frac{v^2}{c^2}}\) & Yields the observed length measured by an observer moving at speed $v$ relative to the object being measured \\

  \notabene{Length contraction takes place only along the direction of motion}

  \tablesubsection{Relativistic Momentum}

  \(p\equiv\displaystyle\frac{mv}{\sqrt{1-\frac{v^2}{c^2}}}=\gamma mv\) & Yields the relativistic momentum of an object where $v\ll c$ \\

  \tablesubsection{Relative Velocity in Special Relativity}

  \(\vec{v}_{AB}=\vec{v}_{AE}-\vec{v}_{BE}\) & Yields relative velocity in Galilean relativity for an object $B$ moving with velocity $\vec{v}_{AE}$ relative to object $E$ where $\vec{v}_{AB}$ is the velocity of the object relative to an independent object $A$ \\
  \(\vec{v}_{AB}=\displaystyle\frac{\vec{v}_{AE}-\vec{v}_{BE}}{1-\displaystyle\frac{\vec{v}_{AE}\vec{v}_{BE}}{c^2}}\) & Yields relative velocity in special relativity for velocities at least \SI{10}{\percent} the speed of light \\
  \(\vec{v}_{AE}=\displaystyle\frac{\vec{v}_{AB}+\vec{v}_{BE}}{1+\displaystyle\frac{\vec{v}_{AB}\vec{v}_{BE}}{c^2}}\) & Yields the relativistic addition of velocities to produce, as in the situation described previously, the velocity measured by an observer at rest relative to object $E$ as the algebraically solved form of the previous equation \\
  \(\vec{v}_{AE}=\displaystyle\frac{\vec{v}_{AB}+\vec{v}_{BE}}{1+\displaystyle\frac{\vec{v}_{AB}\vec{v}_{BE}}{c^2}}=\frac{c+\vec{v}_{BE}}{1+\displaystyle\frac{c\vec{v}_{BE}}{c^2}}=\frac{c\left(1+\displaystyle\frac{\vec{v}_{BE}}{c}\right)}{1+\displaystyle\frac{\vec{v}_{BE}}{c}}=c\) & Yields the relative velocity of a beam of light projected forward from object $B$ in the previously described system as observed from object $E$ \\

  \notabene{The speed of light is the same for all observers}

  \tablesubsection{Relativistic Energy \& The Equivalence of Mass and Energy}

  \(KE=\gamma mc^2-mc^2\) & Yields the kinetic energy of an object with rest energy $mc^2$ \\
  \(E_R=mc^2\) & Yields the rest energy of an object \\
  \(E=KE+E_R=KE+mc^2=\gamma mc^2\) & Yields the total energy $E$ of a system \\
  \(E=\displaystyle\frac{mc^2}{\sqrt{1-\displaystyle\frac{\vec{v}^2}{c^2}}}\) & An alternate form of the total energy equation because $\gamma=\left(1-\frac{\vec{v}^2}{c^2}\right)^{-\frac{1}{2}}$. This is Einstein's famous \textit{mass-energy equivalence equation} \\

  \notabene{Due to $E=\gamma mc^2=KE+mc^2$, a stationary particle with zero kinetic energy has an energy proportional to its mass}

  \tablesubsection{Energy \& Relativistic Momentum}

  \(E^2=p^2c^2+\left(mc^2\right)^2\) & Relates the total energy $E$ to the relativistic momentum $p$. When the particle is at rest, $p=0$ so $E=E_R=mc^2$ \\
  \(E=pc\) & Relates total energy $E$ to relativistic momentum $p$ for mass-less particles such as photons \\
  \(\SI{1}{\electronvolt}=\SI{1.60e-19}{\joule}\) & The conversion factor between the electron volt \si{\electronvolt} and the joule \si{\joule} \\

  \tablesubsection{General Relativity}

  \notabene{Mass determines the inertia of an object and also the strength of the gravitational field. The mass involved in inertia is the inertial mass $m_i$ whereas the mass responsible for the gravitational field is the gravitational mass $m_g$. It appears that gravitational mass and inertial mass may indeed be exactly equal: $m_i=m_g$}
  \multicolumn{2}{c}{
    \begin{tabular}{p{0.5\textwidth} p{0.5\textwidth}}
      Gravitational property & \(\vec{F}_g=G\displaystyle\frac{m_gm_g^\prime}{r^2}\) \\
      Inertial property & \(\vec{F}_i=m_i\vec{a}\)
    \end{tabular}
    }
\end{longtable}
%%% Local Variables:
%%% mode: latex
%%% TeX-master: "../main"
%%% End: