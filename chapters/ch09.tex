\begin{longtable}{p{0.5\textwidth} p{0.5\textwidth}}
  \tablesection{Chapter 9: Solids \& Fluids}\label{sec:ch09}
  \tablesubsection{Density \& Pressure}

  \(\rho\equiv\displaystyle\frac{M}{V}\) & Yields density $\rho$ in \si{\kilo\gram\per\meter\cubed} \\
  \(P\equiv\displaystyle\frac{F}{A}\) & Yields pressure $P$ as force over area in Pascals $\si{\pascal}=\si{\newton\per\meter\squared}$ \\

  \notabene{The specific gravity of a substance is the ratio of its density to the density of water at \SI{4}{\celsius} which is $\SI{1.0e3}{\kilo\gram\per\meter\cubed}$}

  \tablesubsection{Deformation of Solids}

  \(\textrm{stress} = \textrm{elastic modulus} \times \text{strain} \) & For sufficiently small stresses, stress is proportional to strain. This formula is similar to Hooke's Law for springs $\vec{F}=-k\Delta\vec{x}$ \\
  \(\displaystyle\frac{F}{A}=Y\frac{\Delta L}{L_i}\) & Yields tensile strain where $\Delta L$ is the change in length compared to the initial length $L_i$ and $Y$ is Young's modulus \\
  \(\displaystyle\frac{F}{A}=S\frac{\Delta\vec{x}}{h}\) & Yields shear stress where $S$ is the shear modulus $\Delta\vec{x}$ is the distance moved in the plane of the force due to shear stress and $h$ is the height of the object \\
  \(\displaystyle\Delta P=-B\frac{\Delta V}{V_i}\) & Yields volume stress $\Delta P$ where $B$ is the bulk modulus and $\Delta V$ is the change in volume compared to the original volume $V_i$ \\

  \tablesubsection{Variation of Pressure with Depth}

  \(P = P_a + \rho gh\) & Yields pressure $P$ where $P_a$ is the atmospheric pressure $\left(\SI{1.013e5}{\pascal}\right)$ and $g$ is the acceleration due to gravity, and $h$ is the depth below the surface of the fluid \\

  \notabene{Pascal's Principle states that a change in pressure applied to an enclosed fluid is transmitted undiminished to every point of the fluid and to the walls of the container}

  \tablesubsection{Buoyant Forces \& Archimedes' Principle}

  \(B=\rho_{fluid}V_{fluid}g\) & Yields buoyancy of an object where $\rho_{fluid}$ is the density of the fluid it is submerged in $V_{fluid}$ is the volume of displaced fluid and $g$ is the acceleration due to gravity \\

  \notabene{Archimedes' Principle states any object completely or partially submerged in a fluid is buoyed up by a force with magnitude equal to the weight of the fluid displaced by the object}

  \tablesubsection{Fluids in Motion}

  \(A_1\vec{v}_1=A_2\vec{v}_2\) & The equation of continuity, where $A_i$ is the cross-sectional area of a pipe and $\vec{v}_i$ is the fluid speed at point $i$ \\
  \(P + \frac{1}{2}\rho\vec{v}^2 + \rho g\vec{y} = \textrm{constant}\) & Bernoulli's equation, which states the sum of the pressure $P$, the kinetic energy per unit volume $\frac{1}{2}\rho\vec{v}^2$ and the potential energy per unit volume $\rho g\vec{y}$ has the same value at all points along a streamline \\
  \notabene{\textit{Ideal Fluids} are non-viscous, meaning there is no internal friction force between adjacent layers, incompressible, meaning density is constant, move with steady fluid motion, meaning that velocity density and pressure at each point in the fluid don't change with time, and move without turbulence, meaning each element of the fluid has zero angular velocity about its center, so there can't be any eddy currents present in the moving fluid}
  \notabene{Swiftly moving fluids exert less pressure than do slowly moving fluids}
  
  \tablesubsection{Miscellaneous Fluid Dynamics Formul\ae}

  \(\vec{v}_{ex}=\displaystyle\sqrt{\frac{2\left(P-P_{atm}\right)}{\rho}}\) & Yields exhaust speed for a rocket engine \\

  \tablesubsection{Surface Tension, Capillary Action, and Viscous Fluid Flow}

  \(\gamma\equiv\displaystyle\frac{\vec{F}}{L}\) & Yields surface tension $\gamma$ with surface tension force $F$ and length $L$ across which the force acts \\
  \(\abs{\vec{F}}=F=\gamma L=\gamma\left(2\pi r\right)\) & Yields the magnitude of the force of surface tension for a fluid in a cylinder undergoing capillary action \\
  \(\vec{F}_v=\gamma\left(2\pi r\right)\left(\cos\phi\right)\) & Yields the vertical component of the force of surface tension where $\phi$ is the exterior angle between the meniscus and the side of the cylindrical container \\
  \(w=Mg=\rho Vg=\rho g\pi r^2h\) & Condition for equilibrium in a system involving water undergoing capillary action \\
  \(h=\displaystyle\frac{2\gamma}{\rho gr}\cos\phi\) & Yields the height to which a fluid undergoing capillary action is drawn in a cylindrical container \\
  \(F=\eta\displaystyle\frac{A\vec{v}}{d}\) & Yields the magnitude of the force caused by viscous fluid flow where $\eta$ is the coefficient of viscosity in \si{\newton\second\per\meter\squared}, $A$ is the area in contact with the fluid, $\vec{v}$ is the speed of the fluid and $d$ is the distance between the two surfaces between which the viscous fluid is flowing \\
  \(\displaystyle\frac{\Delta V}{\Delta t} = \frac{\pi R^4\left(P_1-P_2\right)}{8\eta L}\) & Poiseuille's Law; yields the rate of flow of a viscous fluid through a section of tube length $L$ and radius $R$ under pressure $P_1$ at one end and $P_2$ at the other \\
  \(RN=\displaystyle\frac{\rho vd}{\eta}\) & Reynolds Number $\left(RN\right)$ where $\rho$ is the density of the fluid, $v$ is the average speed of the fluid along the direction of flow $d$ is the diameter of the tube and $\eta$ is the viscosity of the fluid. If $RN$ is below approximately $2000$, the flow of fluid is streamline; turbulence occurs if $RN$ is above approximately $3000$. In the region between $2000$ and $3000$ flow is unstable, meaning it can move in streamline flow but disturbances can cause turbulent flow \\

  \tablesubsection{Transport Phenomena}

  \(\displaystyle\frac{\Delta M}{\Delta t}=DA\left(\frac{C_2-C_1}{L}\right)\) & Fick's Law where $D$ is a the diffusion coefficient in \si{\meter\squared\per\second} and $A$ is cross-sectional area. The left side of the equation is the \textit{diffusion rate} and $\frac{\left(C_2-C_1\right)}{L}$ is the \textit{concentration gradient} \\
  \(F_r = 6\pi\eta rv\) & Stoke's Law yields the magnitude of the resistive force on a very small spherical object of radius $r$ falling slowly through a fluid of viscosity $\eta$ with speed $v$ \\
  \(w = \rho gV = \rho g\displaystyle\left(\frac{4}{3}\pi r^3\right)\) & Yields the force of gravity acting on a spherical object falling through a viscous fluid \\
  \(\vec{B} = \rho_{fluid}gV = \rho_{fluid}g\displaystyle\left(\frac{4}{3}\pi r^3\right)\) & Yields the buoyant force acting on a spherical object falling through a viscous fluid \\
  \(\vec{v}_t = \displaystyle\frac{2r^2g}{9\eta}\left(\rho-\rho_{fluid}\right)\) & Yields terminal velocity of a spherical object with density $\rho$ and radius $r$ descending through a fluid with density $\rho_{fluid}$ and viscosity $\eta$ with an acceleration due to gravity $g$ \\
  \(\vec{v}_t = \displaystyle\frac{mg}{k}\left(1-\frac{\rho_{fluid}}{\rho}\right)\) & Yields terminal velocity for a non-spherical object of density $\rho$ and mass $m$ descending through a fluid of density $\rho_{fluid}$ where $g$ is the acceleration due to gravity and $k$ is a coefficient which must be determined experimentally via the resistive force $F_r=kv$ \\
  \(\vec{v}_t=\displaystyle\frac{m\vec{\omega}^2r}{k}\left(1-\frac{\rho_{fluid}}{\rho}\right)\) & Yields the sedimentation rate $\vec{v}_t$ (the rate at which particles descend through a fluid) in a centrifuge with angular velocity $\vec{\omega}$ and distance $r$ from point $O$ about which the centrifuge rotates \\
\end{longtable}
%%% Local Variables:
%%% mode: latex
%%% TeX-master: "../main"
%%% End: