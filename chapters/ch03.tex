\begin{longtable}{p{0.5\textwidth} p{0.5\textwidth}}
  \tablesection{Chapter 3: Vectors and Two-Dimensional Motion}  
    
  \tablesubsection{Resultant Vector Formul\ae}
    
  \(\vec{v}_{res} = \displaystyle\sqrt{\left(\sum \vec{v}_x\right)^2 + \left(\sum \vec{v}_y\right)^2}\) & An application of the Pythagorean theorem which yields the magnitude of the resultant velocity $\vec{v}_{res}$ between two or more velocity vectors broken into $x$- and $y$-components. This may be applied to resultant displacement $\Delta\vec{x}_{res}$ and resultant acceleration $\Delta\vec{a}_{res}$ vectors \\
  \(\theta_{res} = \arctan\left(\displaystyle\frac{\sum \vec{v}_y}{\sum \vec{v}_x}\right)\) & The resultant angle $\theta_{res}$ between two or more velocity vectors, broken into $x$- and $y$-components \\
  \(\theta_{opp} = \arctan\left(\displaystyle\frac{\sum \vec{v}_y}{\sum \vec{v}_x}\right) + \SI{180}{\degree}\) & The angle opposite the resultant velocity vector $\theta_{opp}$ \\
	
  \notabene{These same formul\ae\space applied to velocity $\vec{v}$ can be applied to displacement $\Delta\vec{x}$ and to acceleration $\vec{a}$}

  \tablesubsection{Displacement, Velocity, and Acceleration in Two Dimensions}
	
  \(\Delta\vec{r}\equiv\vec{r}_f - \vec{r}_i\) & The displacement $\Delta\vec{r}$ is the change in the position vector of an object \\
  \(\vec{v}_{avg}\equiv\displaystyle\frac{\Delta\vec{r}}{\Delta t}\) & The average velocity $\vec{v}_{avg}$ with displacement $\Delta\vec{r}$ across time interval $\Delta t$ in \si{\meter\per\second} \\
  \(\vec{v}_{ins}\equiv\displaystyle\lim_{\Delta t\to 0}\frac{\Delta\vec{r}}{\Delta t}\) & The instantaneous velocity $\vec{v}$ \\ \\%white space between formulas
  \(\vec{a}_{avg}\equiv\displaystyle\frac{\Delta\vec{v}}{\Delta t}\) & The average acceleration $\vec{a}_{avg}$ with change in velocity $\Delta\vec{f}$ across time interval $\Delta t$ in \si{\meter\per\second\squared} \\
  \(\vec{a}_{ins}\equiv\displaystyle\lim_{\Delta t\to 0}\frac{\Delta\vec{v}}{\Delta t}\) & The instantaneous acceleration $\vec{a}$ \\ \\%adds white space between formulas
  \(R\equiv\Delta x = \displaystyle\frac{2\vec{v}_i\sin\theta_i}{g}\) & The \textit{range} equation; yields the maximum horizontal displacement of a projectile where $y_i = y_f$ and the only acceleration acting on the object is $g$ \\
	
  \tablesubsection{Vector Applications of Polar Conversion Formul\ae}
  \vspace{2mm}
  \begin{tabular}{c c c c}
    \(\Delta x = d\cos\theta\) & \(\vec{v}_x = v\cos\theta\) & \(a_x = a\cos\theta\) & \(F_x = F\cos\theta\) \\
    \(\Delta y = d\sin\theta\) & \(\vec{v}_y = v\sin\theta\) & \(a_y = a\sin\theta\) & \(F_y = F\sin\theta\) \\
  \end{tabular} & \\ \\%white space to force the \tablesubsection{} to fall entirely on the next page

  \tablesubsection{Component Vector Formul\ae}
	
  \begin{tabular}{c c}
    \(\vec{v}_{xf} = \vec{v}_{xi} + \vec{a}_xt\) & \(\Delta x = \vec{v}_{xi}t + \frac{1}{2}\vec{a}_xt^2\) \\
    \(\vec{v}_{yf} = \vec{v}_{yi} + \vec{a}_yt\) & \(\Delta y = \vec{v}_{yi}t + \frac{1}{2}\vec{a}_yt^2\) \\
  \end{tabular} & \\

\end{longtable}
%%% Local Variables:
%%% mode: latex
%%% TeX-master: "../main"
%%% End: