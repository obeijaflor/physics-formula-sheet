\begin{longtable}{p{0.5\textwidth} p{0.5\textwidth}}			
  \tablesubsection{Moment of Inertia Formul\ae}
  \label{ssec:moment-inertia}\\
&\\%needed for spacing
\end{longtable}	
\vspace{-2cm}
\begin{tabular}{l l l}
  Particle: \( I = mr^2 \) & Annular cylinder: \( \displaystyle I = \frac{1}{2}{m}({r_1^2 + r_2^2}) \) & Thin rod (middle): \( \displaystyle I = \frac{1}{12}{m}{r^2} \) \\ 
  Solid sphere: \( \displaystyle I = \frac{2}{5}{m}{r^2} \) & Thin rod (end): \( \displaystyle I = \frac{1}{3}{m}{L^2} \) & Hollow sphere; \( \displaystyle I = \frac{2}{3}{m}{r^2} \) \\
  Hoop or ring: \( I = mr^2 \) & Rectangular plate: \( \displaystyle I = \frac{1}{12}{m}(a^2 + b^2) \) & Cylinder or disk; \( \displaystyle I = \frac{1}{2}{m}{r^2} \) \\
  Thin sheets: see thin rods & & \\
  \bottomrule
\end{tabular}

A table of moments of inertia $I$ for several various shapes. Applications usually are found for torque equations of the form \( \tau = I \alpha \). The moment of inertia is the rotational analogue to mass.
%%% Local Variables:
%%% mode: latex
%%% TeX-master: "../main"
%%% End:

