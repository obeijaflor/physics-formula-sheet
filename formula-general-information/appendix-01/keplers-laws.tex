\begin{longtable}{p{0.5\textwidth} p{0.5\textwidth}}
	\tablesubsection{Kepler's Laws of Planetary Motion}\label{ssec:keplers-laws}
    &\\%this is for spacing of the \tablesubsection{} command
\end{longtable}

\textbf{First Law:} All planets move in elliptical orbits with the Sun at one of the focal points.

\textbf{Second Law:} A line drawn from the Sun to any planet sweeps out equal areas in equal time intervals (i.e., velocity is maximum at perihelion and minimum at aphelion).

\textbf{Third Law:} The square of the orbital period of any planet is proportional to the cube of the average distance from that planet to the Sun, i.e. 

\begin{equation*} 
	\begin{split}
		T^2 &= K_Sr^3 \\
        T^2 &= \displaystyle\left(\frac{4\pi^2}{GM_S}\right)r^3 \\ 
        T   &= \displaystyle\sqrt{\left(\frac{4\pi^2}{GM_S}\right)r^3}.
     \end{split}
\end{equation*}

This yields the period $T$ of a planet where $M_S$ is the mass of the sun. $r$ is the average radius of the planet, and $K_S$ is a constant exactly equal to the quantity \( \displaystyle\frac{4\pi^2}{GM_S} \), or approximately \( \SI{2.97e-19}{\second\squared\per\meter\cubed} \). As a reminder, \( G \) (``big-G'') is the universal gravitation constant approximately equal to \( \SI{6.67384e-11}{kg^{-1} m^3 s^{-2}} \).

For highly eccentric orbits, use $a$, the semi-major axis, instead. A discussion of orbital eccentricity can be found at \url{http://en.wikipedia.org/wiki/Orbital_eccentricity}. This yields 

\begin{equation*} 
	\begin{split}
		T^2 &= a^3 \\
		T &= \displaystyle\sqrt{a^3}.
     \end{split}
\end{equation*}

This equation relates the period of a body to its semi-major axis, which is measured in astronomical units (\si{\astronomicalunit}).

We may also arrive at the mass of the sun, \( M_S \), by moving around variables to arrive at

\[ M_S = \displaystyle\frac{4\pi^2}{GK_S} \approx \SI{1.989e30}{\kilogram}. \]


%%% Local Variables:
%%% mode: latex
%%% TeX-master: "main"
%%% End: