\begin{longtable}{p{0.5\textwidth} p{0.5\textwidth}}
  \tablesection{Chapter 19: Magnetism}
  \tablesubsection{Magnetic Fields}

  \(F=qvB\sin\theta\) & Yields the magnetic force $F$ where $q$ is the charge of the particle, $v$ is the magnitude of the velocity, $B$ is the magnitude of the magnetic field, and $\theta$ is the angle between the direction of $v$ and $B$ \\
  \(F_{max}=qvB\) & Yields the maximum force on a charged particle moving in a magnetic field. This value is attained when $\theta=\SI{90}{\degree}\therefore\sin\theta=1$ \\
  \(B\equiv\displaystyle\frac{F}{qv\sin\theta}\) & Defines the magnitude of the magnetic field, measured in Teslas $\si{\tesla}=\si{\weber\per\meter\squared}=\si{\newton\per\ampere\per\meter}$ \\
  \(\SI{1}{\tesla}=10^4\,\si{\gauss}\) & Relates teslas \si{\tesla} to gauss \si{\gauss} \\

  \notabene{A stationary charged particle does not interact with a static magnetic field. When a charged particle is \textit{moving} through a magnetic field, however, a magnetic force acts on it}

  \tablesubsection{Magnetic Force on a Current-Carrying Conductor}

  \(F=BI\ell\sin\theta\) & Yields the force on a current-carrying conductor in a segment of wire with length $\ell$, current $I$, magnitude of the magnetic field $B$ and where $\theta$ is the angle between the direction of current flow and $B$ \\

  \tablesubsection{Torque on a Current Loop \& Electric Motors}

  \(\tau_{max}=BIA\) & Yields the maximum torque operating on a loop in a uniform magnetic field where $B$ is the magnitude of the magnetic field, $I$ is the current flowing through the loop and $A$ is the cross-sectional area of the loop. This formula is only valid when the magnetic field is \textit{parallel} to the plane of the loop \\
  \(\tau=BIA\sin\theta\) & Yields the torque operating on a loop in a uniform magnetic field where the magnetic field makes an angle $\theta$ with a line perpendicular to the plane of the loop \\
  \(\tau=BIAN\sin\theta=\mu B\sin\theta\) & Yields the torque operating on a coiled loop in a uniform magnetic field where $N$ is the number of coils where $\mu=IAN$ is the magnetic moment of the coil \\

  \tablesubsection{Motion of a Charged Particle in a Magnetic Field}

  \(F=qvB=\displaystyle\frac{mv^2}{r}\) & Yields the magnetic force acting on a charged particle in a magnetic field with centripetal acceleration $\frac{v^2}{r}$ \\
  \(r=\displaystyle\frac{mv}{qB}\) & Yields the radius of motion of a charged particle in a magnetic field. Additionally, this equation states that the radius $r\propto mv$ the momentum of the particle and is inversely proportional to the charge $q$ and the magnetic field $B$. This equation is often known as the \textit{cyclotron equation} because it is used in the design of these instruments (\href{http://en.wikipedia.org/wiki/CERN}{CERN is an example of a cyclotron}) \\

  \notabene{The magnetic force is always directed toward the center of the circular path}

  \tablesubsection{Magnetic Field of a Long, Straight Wire \& Amp\`ere's Law}

  \(B=\displaystyle\frac{\mu_0I}{2\pi r}\) & Yields the magnetic field due to a long, straight wire where $\mu_0=4\pi\e{-7}\,\si{\tesla\meter\per\ampere}$ is the permeability of free space, $I$ is the current flowing through the wire, and $r$ is the radius of the magnetic field $B$ \\
  \(\displaystyle\sum B_{\parallel}\Delta\ell=\mu_0I\) & \textit{Amp\`ere's Circuital Law}, where $B_{\parallel}$ is the component of the magnetic field parallel ($\parallel$) to the segment of a path $\ell$ with length $\Delta\ell$. The sum of all products $B_{\parallel}\Delta\ell$ is equal to $\mu_0$ times the net current $I$ that passes through the surface bounded by the closed path \\

  \notabene{The equation for Amp\`ere's Circuital Law can be rearranged as \(\sum B_{\parallel}\Delta\ell=B_{\parallel}\sum\Delta\ell=B\left(2\pi r\right)=\mu_0I\rightarrow B=\frac{\mu_0I}{2\pi r}\)}

  \tablesubsection{Magnetic Force Between Two Parallel Conductors}

  \(F_1=B_2I_1\ell=\left(\displaystyle\frac{\mu_0I_2}{2\pi d}\right)I_1\ell=\displaystyle\frac{\mu_0I_1I_2\ell}{2\pi d}\) & In a system of two long, straight, parallel wires separated by distance $d$ and carrying currents $I_1$ and $I_2$ in the same direction exerting magnetic fields $B_1$ and $B_2$ where the length of wire considered is $\ell$, $B_2=\frac{\mu_0I_2}{2\pi d}$, $F_1$ yields the magnetic force acting on wire 1 in the presence of field $B_2$ due to $I_2$, which can be written in terms of the force per unit length $\left(\frac{F_1}{\ell}\right)$ \\
  \(\displaystyle\frac{F_1}{\ell}=\frac{\mu_0I_1I_2}{2\pi d}\) & An alternate form of the previous equation \\

  \notabene{Parallel conductors carrying currents in the same direction \textit{attract} one another while parallel conductors carrying currents in opposite directions \textit{repel} one another}
  \notabene{\textit{Definition of the Ampere} If two long, parallel wires \SI{1}{\meter} apart carry the same current and the magnetic force per unit length on each wire is $\SI{2e-7}{\newton\per\meter}$, the current is defined to be \SI{1}{\ampere}}
  \notabene{\underline{Definition of the Coulomb} If a conductor carries a steady current of \SI{1}{\ampere}, the quantity of the charge that flows through any cross section in \SI{1}{\second} is \SI{1}{\coulomb}}

  \tablesubsection{Magnetic Fields of Current Loops \& Solenoids}

  \(B=\displaystyle\frac{\mu_0I}{2R}\) & Yields the magnitude of the magnetic field at the center of a circular loop of radius $R$ carrying $I$ \\
  \(B=\mu_0nI\) & Yields the magnitude of the magnetic field inside a solenoid (electromagnet) where $n=\frac{N}{\ell}$ is the number of turns per unit length of the solenoid \\

\end{longtable}
%%% Local Variables:
%%% mode: latex
%%% TeX-master: "main"
%%% End: