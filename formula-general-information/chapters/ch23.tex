\begin{longtable}{p{0.5\textwidth} p{0.5\textwidth}}
  \tablesection{Chapter 23: Mirrors \& Lenses}
  \tablesubsection{Flat Mirrors}

  \(M\equiv\displaystyle\frac{\textrm{image height}}{\textrm{object height}}=\frac{h^\prime}{h}\) & Yields the lateral magnification $M$ of a mirror with image height $h^\prime$ and object height $h$. For a flat mirror $M=1$ because $h^\prime=h$ \\

  \notabene{The image formed by an object placed in front of a flat mirror is as far behind the mirror as the object is in front of the mirror. The same holds true for height}
  \notabene{Images are formed at the point where rays of light actually intersect or where they appear to originate}

  \tablesubsection{Concave Mirrors}

  \(M=\displaystyle\frac{h^\prime}{h}=-\frac{q}{p}\) & Yields the lateral magnification of a concave mirror where $q$ is the image distance, the distance of the image behind the mirror, and $p$ is the object distance, the distance of the object in front of the mirror \\
  \(\displaystyle\frac{1}{p}+\frac{1}{q}=\frac{2}{R}\) & The Mirror Equation, where $R$ is the radius of curvature of the mirror (the radius of the circle formed by the curvature of the mirror) \\
  \(f=\displaystyle\frac{R}{2}\) & Yields the focal length $f$ \\
  \(\displaystyle\frac{1}{p}+\frac{1}{q}=\frac{1}{f}\) & Yields the mirror equation in terms of the focal length \\

  \notabene{If an object is very far from a mirror|if the object distance $p$ is great enough compared with $R$ that $p$ can be said to approach infinity|then $\frac{1}{p}\approx 0$ and $q\approx\frac{R}{2}$. In other words, when an object is very far from the mirror, the image point is halfway between the center of curvature and the center of the mirror, because the incoming rays of light are essentially parallel. In this instance, we call the image point the \textit{focal point} $F$ and the image distance the \textit{focal length} $f$}
  
  \notabene{Rays from objects at infinity are always focused at the focal point}
  
  \tablesubsection{Images Formed by Refraction}
  
  \(\displaystyle\frac{n_1}{p}+\frac{n_2}{q}=\frac{n_2-n_1}{R}\) & Snell's law of refraction applied to two media with indices of refraction $n_1$ and $n_2$ where the boundary between them is spherical. It is assumed that $n_2>n_1$ \\
  \(M=\displaystyle\frac{h^\prime}{h}=-\frac{n_1q}{n_2p}\) & Yields the magnification $M$ of a refracting surface \\
  
  \(\displaystyle\frac{n_1}{p}=-\frac{n_2}{q} \therefore q=-\frac{n_2}{n_1}p\) & Yields the reduction of Snell's law of refraction as $R$ approaches infinity \\
  
  \notabene{In contrast to mirrors, real images in lenses are formed by refraction on the side of the surface \textit{opposite} the side from which the light comes}
  \notabene{The image formed by a flat refracting surface is on the same side of the surface as the object}
  
  \tablesubsection{Thin Lenses}
  
  \(M=\displaystyle\frac{h^\prime}{h}=-\frac{q}{p}\) & The equation for magnification $M$ of a lens is the same as for a mirror \\
  \(\displaystyle\frac{1}{p}+\frac{1}{q}=\frac{1}{f}\) & \textit{The Thin Lens Equation} which can be applied to both diverging and converging lenses \\
  \(\displaystyle\frac{1}{f}=\left(n-1\right)\left(\frac{1}{R_1}-\frac{1}{R_2}\right)\) & The \textit{Lens-maker's equation} where $f$ is the focal point of the lens, $R_1$ and $R_2$ are the radii of curvature of each side of the lens and $n$ is the index of refraction of the lens material \\

  \notabene{Rays parallel to the principal axis diverge after passing through a lens of biconcave shape. A \textit{converging} lens causes the rays to converge upon a focal point while a \textit{diverging} lens causes the rays to diverge}
  \notabene{A converging lens has a positive focal length and a diverging lens has a negative focal length, as yielded by the thin lens equation}
\end{longtable}
%%% Local Variables:
%%% mode: latex
%%% TeX-master: "main"
%%% End: