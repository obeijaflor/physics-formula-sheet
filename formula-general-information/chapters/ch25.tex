\begin{longtable}{p{0.5\textwidth} p{0.5\textwidth}}
  \tablesection{Chapter 25: Optical Instruments}
  \tablesubsection{The Camera}

  \(f\textrm{-number}\equiv\displaystyle\frac{f}{D}\) & Yields the $f$-number of a lens, a description of lens ``speed,'' of a lens with lens diameter $D$ and focal length $f$ \\
  \(P=\displaystyle\frac{1}{f}\) & Yields the power $P$ of a lens in diopters with focal length $f$ \\

  \notabene{A lens with a low $f$-number is a ``fast'' lens}

  \tablesubsection{The Simple Magnifier}

  \(m\equiv\displaystyle\frac{\theta}{\theta_o}\) & Yields the angular magnification $m$ of a lens where $\theta$ is the angle subtended by a small object when the lens is in use and $\theta_o$ is the angle subtended by the object placed at the near point $p$ with no lens in use \\

  \notabene{The object distance $q$ of an object at the focal point of the eye is $q=\SI{-25}{\centi\meter}$}

  \tablesubsection{The Compound Microscope}

  \(M_1=-\displaystyle\frac{q_1}{p_1}\approx -\frac{L}{f_o}\) & Yields the lateral magnification of the objective lens of a compound microscope with an objective lens of very short focal length $f_o<\SI{1}{\centi\meter}$ and an eyepiece of focal length $f_e>\SI{1}{\centi\meter}$ with distance $L$ separating them where $L\gg f_0$ and $L\gg f_e$ \\
  \(m_e=\displaystyle\frac{\SI{25}{\centi\meter}}{f_e}\) & Yields the angular magnification $m_e$ of the eyepiece for an object placed at the focal point \\
  \(m=M_1m_e=-\displaystyle\frac{L}{f_0}\left(\frac{\SI{25}{\centi\meter}}{f_e}\right)\) & Yields the overall magnification $m$ of the compound microscope with lateral magnification $M_1$ and angular magnification $m_e$ \\

  \tablesubsection{The Telescope}

  \(m=\displaystyle\frac{\theta}{\theta_o}=\frac{\frac{h^\prime}{f_e}}{\frac{h^\prime}{f_o}}=\frac{f_o}{f_e}\) & Yields the angular magnification of a telescope \\

  \tablesubsection{Resolution of Single-Slit \& Circular Apertures}

  \(\theta_{min}\approx\displaystyle\frac{\lambda}{a}\) & Yields the limiting angle for a slit which satisfies Rayleigh's criterion. Because $\lambda\ll a$ in most situations, $\sin\theta\approx\theta$ for a slit of width $a$ \\
  \(\theta_{min}=1.22\displaystyle\frac{\lambda}{D}\) & Yields the limiting angle of resolution for a circular aperture with diameter $D$ \\
  \(R=Nm\) & Yields the resolving power of a diffraction grating where $N$ is the number of lines of the grating which are illuminated and $m$ is the order number of the grating \\

  \notabene{When the central maximum of one image falls on the first minimum of another image, the images are said to be just resolved. This limiting condition of resolution is known as \textit{Rayleigh's Criterion}}
  \notabene{For $m=0$,$R=0$, which signifies that \textit{all wavelengths are indistinguishable} for the zeroth-order maximum}
\end{longtable}
%%% Local Variables:
%%% mode: latex
%%% TeX-master: "main"
%%% End: