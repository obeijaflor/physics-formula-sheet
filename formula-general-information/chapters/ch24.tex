\begin{longtable}{p{0.5\textwidth} p{0.5\textwidth}}
  \tablesection{Chapter 24: Wave Optics}
  \tablesubsection{Young's Double-Slit Experiment}

  \(\delta=r_2-r_1=d\sin\theta\) & Yields the path difference $\delta$ in an experiment where we consider point $P$ on a viewing screen positioned a perpendicular distance $L$ from a screen containing slits $S_1$ and $S_2$ which are separated by distance $d$ and $r_1$ and $r_2$ are the distances the secondary waves travel from slit to screen (the primary wave travels through $S_0$, the single slit in the first screen, to the slits) where the light intensity on the screen at $P$ is the result of light from both slits \\

  \begin{tabular}{l l}
    \(\delta=d\sin\theta_{\textrm{bright}}=m\lambda\) & \(m=0,\pm 1,\pm 2,\ldots\)
  \end{tabular} & The condition for constructive interference at point $P$ where $m$ is the \textit{order number} and $\lambda$ is the wavelength of the wave. At each maximum, $\theta_{\textrm{brignt}}$ is either zero or some integral multiple of the wavelength \\
  \begin{tabular}{l l}
    \(\delta=d\sin\theta_{\textrm{dark}}=\left(m+\frac{1}{2}\right)\lambda\) & \(m=0,\pm 1,\pm 2,\ldots\)
  \end{tabular} & The condition for destructive interference at point $P$. At each minimum, $\theta_{\textrm{dark}}$ is some odd multiple of $\frac{\lambda}{2}$ and the two waves arriving at $P$ are \SI{180}{\degree} out of phase \\
  \begin{tabular}{l l}
    \(y_{\textrm{bright}}=\displaystyle\frac{\lambda L}{d}m\) & \(m=0,\pm 1,\pm 2,\ldots\)
  \end{tabular} & Yields the positions for bright fringes where $\theta$ is the angle between the midpoint $Q$ between $S_1$ and $S_2$ and $L$ is the distance between $Q$ and the point $O$ immediately opposite $Q$ on the viewing screen \\
  \begin{tabular}{l l}
    \(y_{\textrm{dark}}=\displaystyle\frac{\lambda L}{d}\left(m+\frac{1}{2}\right)\) & \(m=0,\pm 1,\pm 2,\ldots\)
  \end{tabular} & Yields the positions for dark fringes in the system described above \\

  \notabene{The \textit{path difference} $\delta$ is the difference in the distance traveled between the secondary waves in the experiment}
  \notabene{\textit{Constructive interference} occurs when the two waves are in phase (i.e., they have the same amplitude at a particular time)}
  \notabene{\textit{Destructive interference} occurs when the two waves are out of phase (i.e., they do not have the same amplitudes at a particular time)}
  \notabene{The central bright \textit{fringe} (a band that is either bright or dark produced in a double-slit experiment) at $\theta_{\textrm{bright}}=0\space\left(m=0\right)$ is the zeroth-order maximum. The first maximum on either side, where $m\pm 1$ is the first-order maximum, and so forth}

  \tablesubsection{Interference in Thin Films}

  \(\lambda_n=\displaystyle\frac{\lambda}{n}\) & Yields the wavelength of light $\lambda_n$ in a medium with index of refraction $n$ where $\lambda$ is the wavelength of light in vacuum \\
  \begin{tabular}{l l}
    \(2t=\left(m+\frac{1}{2}\right)\lambda_n\) & \(m=0,1,2,\ldots\)
  \end{tabular} & Yields the general form of constructive interference for thin films of uniform thickness $t$ \\
  \begin{tabular}{l l}
    \(2nt=\left(m+\frac{1}{2}\right)\lambda\) & \(m=0,1,2,\ldots\)
  \end{tabular} & An alternate form of the previous equation due to $\lambda_n=\frac{\lambda}{n}$ \\
  \begin{tabular}{l l}
    \(2nt=m\lambda\) & \(m=0,1,2,\ldots\)
  \end{tabular} & Yields the general form of destructive interference for thin films of uniform thickness $t$ \\

  \notabene{An electromagnetic wave traveling between mediums with indices of refraction $n_1$ and $n_2$ undergoes a \SI{180}{\degree} phase change upon reflection when $n_2>n_1$. There is no phase change in the reflected wave if $n_2<n_1$}
  \notabene{The general forms for constructive \& destructive interference with thin films are valid only when there is a single phase reversal}
  \notabene{If the film of uniform thickness $t$ is placed between two different media, one of lower refractive index than the film and the other of higher refractive index, the general forms for constructive \& destructive interference are reversed (e.g., $2nt=m\lambda$ would yield constructive interference)}

  \tablesubsection{Single-Slit Diffraction}

  \begin{tabular}{l l}
    \(\sin\theta_{\textrm{dark}}=m\frac{\lambda}{a}\) & \(m=\pm 1,\pm 2,\pm 3, \ldots\)
  \end{tabular} & The general form of the condition for destructive interference for a single slit of width $a$ where $m$ is the order number \\

  \notabene{According to Huygens' principle, each portion of a slit acts as a source of waves. Hence, light from one portion of the slit can interfere with light from another portion and the resultant intensity on point $P$ depends on the direction of $\theta$}

  \tablesubsection{The Diffraction Grating}

  \begin{tabular}{l l}
    \(d\sin\theta_{\textrm{bright}}=m\lambda\) & \(m=0,\pm 1,\pm 2,\ldots\)
  \end{tabular} & The condition for maxima in the interference pattern of a diffraction grating where $\theta$ is the angle of deviation caused by the grating, $m$ is the order number of the diffraction pattern, and the path difference $\delta=d\sin\theta$ between waves from any two adjacent slits. If this path difference equals one wavelength or some integral multiple of a wavelength, waves from all slits will be in phase at $P$ and a bright line will be observed at that point \\

  \tablesubsection{Polarization of Light Waves}

  \(I=I_0\cos^2\theta\) & \textit{Malus's Law}; yields the intensity $I$ of a polarized light beam transmitted through an analyzer where $I_0$ is the intensity of the polarized wave incident on the analyzer. This applies to any two polarizing materials having transmission axes at an angle $\theta$ to each other \\
  \(n=\displaystyle\frac{\sin\theta_p}{\cos\theta_p}=\tan\theta_p\) & \textit{Brewster's Law}; yields the index of reflection $n$ for a polarizing material with polarizing angle $\theta_p$ \\

  \notabene{A polarizing material reduces the intensity of incoming light by absorbing light having an electric field vector $\vec{E}$ perpendicular to the \textit{transmission axis}|the direction perpendicular to the molecular chains of the polarizing material| and transmit light with an electric field vector parallel to the transmission axis}
  \notabene{A beam of light can be polarized by reflection (see Brewster's Law) whereby the light reflected from a surface (e.g., water) is partially polarized}
\end{longtable}
%%% Local Variables:
%%% mode: latex
%%% TeX-master: "main"
%%% End: