\begin{longtable}{l c}
	\tablesection{Chapter 1: General Formul\ae}

	\notabene{The symbol $\therefore$ is occasionally used in this document, and is a mathematical mark meaning ``therefore''. Additionally, there is a frequent use of the Greek alphabet. See \textit{Appendix II} on page \pageref{ssec:greek-alphabet} for a list of Greek alphabetical characters. For a diagram of the unit circle|a source of useful trigonometric information|see \textit{Appendix II} on page \pageref{ssec:unit-circle}} 

	\notabene{The \textbf{right-hand rule} is useful in determining arbitrary direction and axes. Wrap your open right hand around the object in the direction of its rotation; the direction indicated by your upwardly-pointing thumb may be considered north or the positive direction. You will see this rule a lot in your studies of physics in various applications, specifically with regards to torques and electromagnetism.}
    
	\tablesubsection{Trigonometric Formul\ae}

	\begin{tabular}{c c c}
		\(\sin\theta = \frac{opp}{hyp}\) & \(\cos\theta = \frac{adj}{hyp}\) & \(\tan\theta = \frac{opp}{adj}\) \\
		\(\csc\theta = \frac{hyp}{opp}\) & \(\sec\theta = \frac{hyp}{adj}\) & \(\tan\theta = \frac{adj}{opp}\) \\
	\end{tabular} & Basic Trigonometric Functions \\
	\begin{tabular}{c c c }
		\(\theta = \arcsin\left(\frac{opp}{hyp}\right)\) & \(\theta = \arccos\left(\frac{adj}{hyp}\right)\) & \(\theta = \arctan\left(\frac{opp}{adj}\right)\) \\
		\(\theta = \arccsc\left(\frac{hyp}{opp}\right)\) & \(\theta = \arcsec\left(\frac{hyp}{adj}\right)\) & \(\theta = \arccot\left(\frac{adj}{opp}\right)\) \\
	\end{tabular} & Inverse Trigonometric Functions \\
    
    \notabene{The inverse trigonometric functions are determined by reflecting the graphs of the basic trigonometric functions over the line $y=x$. In this manner, $y=\sin x$ becomes $x=\sin y$ which becomes $y=\arcsin x$}
	
	\tablesubsection{Polar to Cartesian Formul\ae}
	
	\(x = r\cos\theta\) & \(y = r\sin\theta\) \\
\end{longtable}
%%% Local Variables:
%%% mode: latex
%%% TeX-master: "main"
%%% End: