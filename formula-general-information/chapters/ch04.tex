\begin{longtable}{p{0.5\textwidth} p{0.5\textwidth}}
  \tablesection{Chapter 4: The Laws of Motion}
  
  \notabene{For a breakdown of Newton's Laws of Motion, please refer to {\it Appendix I} on page \pageref{ssec:newtons-laws}.}
  
  \tablesubsection{General Motion Formul\ae}

  \(F_g = G\displaystyle\frac{m_1m_2}{r^2}\) & The magnitude of the gravitational force $F_g$ where $G = \SI{6.67e-11}{\newton\meter\squared\per\kilo\gram\squared}$ is the universal gravitation constant, and $r$ is the distance between the two objects with masses $m_1$ and $m_2$ \\
  \(w = mg\) & Weight as a result of the interaction between mass and gravity $g=\SI{9.81}{\meter\per\second\squared}$ at sea level on the Earth's surface \\
  \(g = G\displaystyle\frac{M_E}{r^2}\) & Yields the acceleration $g$ due to gravity at distance $r$ from the center of the Earth \\

  \notabene{For equations involving the calculation of Lagrange points, see \textit{Appendix I} on page \pageref{fig:lagrange_diag}}

  \tablesubsection{Forces}

  \(\sum\vec{F} = m\vec{a}\) & Newton's Second Law; the sum of forces $\vec{F}$ acting on an object with mass $m$ experiencing acceleration $\vec{a}$, measured in Newtons $\si{\newton}=\si{\kilo\gram\meter\per\second\squared}$ \\
  \begin{tabular}{c c c}
    \(\sum F_x = ma_x\) & \(\sum F_y = ma_y\) & \(\sum F_z = ma_z\)
  \end{tabular} & Component force equations \\
  \(\vec{n} = -\vec{F}_g\) & The normal force $\vec{n}$ is equal and opposite to the force of gravity $\vec{F}_g$ \\

  \tablesubsection{Objects in Equilibrium}

  \(\sum\vec{F} = 0\) & Objects that are either at rest or moving with constant velocity are said to be in equilibrium, because $\vec{a} = 0$ \\
  \begin{tabular}{c c}
    \(\sum\vec{F}_x=0\) & \(\sum\vec{F}_y=0\) \\
  \end{tabular} & Component vector sums of forces acting upon objects in equilibrium \\
  \(\vec{f}_s = -\vec{F}\) & For objects in equilibrium, the magnitude of the force of static friction $\vec{f}_s$ is equivalent and opposite to a force acting upon the object \\

  \notabene{For information concerning Atwood devices, see \textit{Appendix I} on page \pageref{ssec:atwood}}

  \tablesubsection{Friction}

  \(f_s \leq \mu_sn\) & Yields the magnitude of the force of static friction $f_s$ between any two surfaces in contact, where $\mu_s$ is the coefficient of static friction and $n$ is the normal force \\
  \(f_k = \mu_kn\) & Yields the magnitude of the force of kinetic friction $f_k$ acting between two surfaces, where $\mu_k$ is the coefficient of kinetic friction and $n$ is the normal force \\
\end{longtable}
%%% Local Variables:
%%% mode: latex
%%% TeX-master: "main"
%%% End: