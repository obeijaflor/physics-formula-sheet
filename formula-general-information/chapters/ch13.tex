\begin{longtable}{p{0.5\textwidth} p{0.5\textwidth}}
  \tablesection{Chapter 13: Vibrations \& Waves}
  \tablesubsection{Simple Harmonic Motion}

  \(F_s=-kx\) & Hooke's Law where $k$ is the spring constant and $x$ is the displacement of the spring from equilibrium. Because Hooke's Law is a restoring force (in that it always pushes or pulls the object toward the equilibrium position) it can be used to describe simple harmonic motion \\
  \(a = -\displaystyle\frac{k}{m}x = -\omega^2x\) & Yields the acceleration of an object moving with simple harmonic motion \\
  \(\displaystyle a_{max}=\frac{k}{m}A=\omega^2A\) & Yields the maximum acceleration for an object in SHM where $A$ is the amplitude|the maximum distance of the object from its equilibrium position where $x=\pm A$ \\
  \(\displaystyle v = \pm\sqrt{\frac{k}{m}\left(A^2-x^2\right)} = \pm\omega\sqrt{A^2-x^2}\) & Yields the velocity of an object moving with simple harmonic motion \\
  \(\displaystyle v_{max} = \pm\sqrt{\frac{k}{m}A^2} = \pm\omega\sqrt{A^2}\) & Yields the maximum velocity for an object in SHM where $x=0$ \\

  \tablesubsection{Elastic Potential Energy}

  \(PE_s\equiv\frac{1}{2}kx^2\) & Yields elastic potential energy \\
  \(\left(KE + PE_g + PE_s\right)_i = \left(KE + PE_g + PE_s\right)_f\) & The Law of Conservation of Energy for springs \\
  \(W_{nc} = \left(KE + PE_g + PE_s\right)_f - \left(KE + PE_g PE_s\right)_i\) & Yields the change in mechanical energy when nonconservative forces are present \\
  \(E_{mech} = \frac{1}{2}kA^2 = \frac{1}{2}mv^2 + \frac{1}{2}kx^2\) & Yields the total mechanical energy of an object undergoing periodic motion \\

 \tablesubsection{Period \& Frequency}

 \(v_0 = \displaystyle\frac{2\pi A}{T}\) & Yields the constant velocity $v_0$ of an object around a circular path where $T$ is the period \\
 \(T = 2\pi\displaystyle\sqrt{\frac{m}{k}}\) & Yields the period $T$ in \si{\second} of an object in SHM on a spring \\
 \(f = \frac{1}{T}\) & Yields the frequency $f$ in \si{\hertz} Hertz of an object in SHM \\
 \(f = \displaystyle\frac{1}{2\pi}\sqrt{\frac{k}{m}}\) & Yields the frequency of an object in SHM on a spring \\
 \(\omega = 2\pi f = \displaystyle\sqrt{\frac{k}{m}}\) & Yields the angular frequency $\omega$ of an object in SHM \\

 \tablesubsection{Position, Velocity, \& Acceleration as a Function of Time}

 \(x = A\cos\left(2\pi ft\right)\) & Yields the $x$-position of an object moving in SHM \\
 \(v = -A\omega\sin\left(2\pi ft\right)\) & Yields the velocity of an object moving in SHM \\
 \(a = -A\omega^2\cos\left(2\pi ft\right)\) & Yields the acceleration of an object moving in SHM \\

 \tablesubsection{Motion of a Pendulum}

 \(F_t = -mg\sin\theta = -mg\sin\displaystyle\left(\frac{s}{L}\right)\) & Yields the force acting tangent to the circular arc of the pendulum where $s$ is the displacement of the pendulum from equilibrium and $L$ is the length of the pendulum \\
 \(T = 2\pi\displaystyle\sqrt{\frac{L}{g}}\) & Yields the period of a pendulum \\
 \(T = 2\pi\displaystyle\sqrt{\frac{I}{mgL}} = 2\pi\sqrt{\frac{L}{g}}\) & Yields the period of a \textit{physical pendulum}, a pendulum of an object of any shape (e.g., a potato) which pivots about point $O$ which is a distance $L$ from the object's center of mass where $I=ml^2$ \\

 \tablesubsection{Waves}

 \(v=f\lambda=\displaystyle\frac{\lambda}{T}\) & Yields the velocity of a wave where $\lambda$ is the wavelength \\
 \(v = \displaystyle\sqrt{\frac{F}{\mu}}\) & Yields the velocity of a wave moving along a string where $F$ is the tension of the string and $\mu$ is the mass per unit length of the string \\
 
\end{longtable}
%%% Local Variables:
%%% mode: latex
%%% TeX-master: "main"
%%% End: