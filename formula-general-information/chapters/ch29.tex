\begin{longtable}{p{0.5\textwidth} p{0.5\textwidth}}
  \tablesection{Chapter 29: Nuclear Physics}
  \tablesubsection{Nucleic Properties}
  
  \notabene{See \textit{Appendix II} on page \pageref{ssec:atomic-mass} for data on the masses of selected subatomic particles}

  \(E_R=mc^2=\left(\SI{1.660559e-27}{\kilo\gram}\right)\left(\SI{2.99792e8}{\meter\per\second}\right)=\SI{1.49231e-10}{\joule}=\SI{931.494}{\mega\electronvolt}=\SI{1}{\atomicmassunit}\) & Yields the energy equivalent of one atomic mass unit \si{\atomicmassunit} \\
  \(r=r_0\displaystyle A^{\frac{1}{3}}\) & Yields the average radius of an atom with mass number $A$ where $r_0=\SI{1.2e-15}{\meter}$. This equation suggests that all nuclei have nearly the same density \\

  \tablesubsection{Radioactivity}

  \(\displaystyle\frac{\Delta N}{\Delta t}\propto N\) & If a radioactive sample contains $N$ radioactive nuclei at some instant, the number of nuclei $\Delta N$ that decay in a small time interval $\Delta t$ is proportional to $N$ \\
  \(\Delta N=-\lambda N\Delta t\) & Yields the number of nuclei $\Delta N$ that decay in a small time interval $\Delta t$ where $N$ is the number of radioactive nuclei at some instant where $\lambda$ is the \textit{decay constant} \\
  \(R=\displaystyle\abs{\frac{\Delta N}{\Delta t}}=\lambda N\) & Yields the decay rate (or activity) $R$ of a sample as the number of decays per second, measured in curies \si{\curie} defined as $\SI{1}{\curie}=\SI{3.7e10}{decays\per\second}$ \\
  \(N=N_0e^{-\lambda t}\) & Yields the number of nuclei present as a function varying with time as a form of the previous equation found with calculus where $N$ is the number of radioactive nuclei present at time $t$ and $N_0$ is the number of nuclei at time $t=0$ and $e=2.718\ldots$ is Euler's constant. Processes which obey this equation are said to undergo \textit{exponential decay} \\
  \(N=N_0\displaystyle\left(\frac{1}{2}\right)^n\) & Yields the number of radioactive nuclei $N$ after $n$ half-lives have occurred. One half-life $T_{\frac{1}{2}}$ is the time it takes for half of a given number of radioactive nuclei to decay \\
  \(n=\displaystyle\frac{t}{T_{\frac{1}{2}}}\) & Relates the number of half-lives passed $n$ during time $t$ to the length of each half-life $T_{\frac{1}{2}}$ \\
  \(T_{\frac{1}{2}}=\displaystyle\frac{\ln 2}{\lambda}=\frac{0.693}{\lambda}\) & Relates the decay constant $\lambda$ to the length of each half-life $T_{\frac{1}{2}}$ \\

  \notabene{\SI{1}{\curie} is the approximate activity of \SI{1}{\gram} of radium. The SI unit of activity is the becquerel \si{\becquerel} defined as $\SI{1}{\becquerel}=\SI{1}{decay\per\second}$}

  \tablesubsection{Alpha Decay}

  \ce{^A_ZX -> ^{A - 4}_{Z - 2}Y + ^4_2He} & Yields the $\alpha$-decay of a particle where X is the parent nucleus and Y is the daughter nucleus \\

  \notabene{If a nucleus emits an $\alpha$ particle (\ce{^4_2He}), it loses two protons and two neutrons. Thus, the neutron number $N$ of a single nucleus decreases by 2, Z decreases by 2, and A decreases by 4}

  \tablesubsection{Beta Decay}

  \(\ce{^A_ZX -> ^A_{Z + 1}Y + e- + \bar{\nu}}\) & $\beta$-decay for this nucleus produces an electron \ce{e-} and an anti-neutrino $\bar{\nu}$ \\
  \(\ce{^A_ZX -> ^A_{Z - 1}Y + e+ + \nu}\) & $\beta$-decay for this nucleus produces a proton \ce{e+} and a neutrino $\nu$ \\
  \(\ce{^1_0n -> ^1_1p + e-}\) & The process by which a neutron $n$ becomes a proton $p$ and an electron $e$ during $\beta$-decay \\

  \notabene{If a nucleus undergoes $\beta$-decay, the daughter nucleus has the same number of nucleons as the parent nucleus, but the atomic number is changed by 1. During $\beta$-decay, either an electron and an anti-neutrino are emitted or a positron and a neutrino are emitted}

  \tablesubsection{Gamma Decay}

  \(\ce{^{12}_5B -> ^{12}_6C\textrm{*} + e- + \bar{\nu}}\) &\\
  \(\ce{^{12}_6C\textrm{*} -> ^{12}_6C + \gamma}\) & A typical example of the $\gamma$-decay process. Note that in this example the \ce{^{12}_6C} from the $\beta$-decay of \ce{^{12}_5B} is in an excited energy state, as indicated by the asterisk \\

  \notabene{If a nucleus exists in an excessively excited state|for example, as a result of a collision or $\alpha$- or $\beta$-decay|it can release a high-energy photon in the form of a $\gamma$-ray, thus decreasing the energy state of the nucleus. Note that $\gamma$-emission does not result in a change in either $Z$ or $A$}

  \tablesubsection{Nuclear Reactions}

  \(KE_{min}=\displaystyle\left(1+\frac{m}{M}\right)\abs{Q}\) & Yields the threshold energy|the minimum kinetic energy|$KE_{min}$ of an incoming particle required to satisfy the energy-balancing requirements in a chemical equation (for example, \ce{^4_2He + ^{14}_7N -> ^{17}_8O + ^1_1H} requires energy on the left side of the equation, thus it is \textit{endothermic}) where $m$ is the mass of the incident particle, $M$ is the mass of the target and $Q$ is the energy either inserted into or released by the system, measured in \si{\mega\electronvolt} \\

  \notabene{\textit{Endothermic} reactions will not occur unless energy is injected into the system while \textit{exothermic} can occur spontaneously and will release energy}
  \notabene{Reactions with negative $Q$-values are endothermic and reactions with positive $Q$-values are exothermic}
\end{longtable}
%%% Local Variables:
%%% mode: latex
%%% TeX-master: "main"
%%% End: